
\chapter*{\Large{\begin{hindi}संक्षेप\end{hindi}}}
\begin{hindi}
यह एक उदाहरणात्मक \begin{english}(Example)\end{english} पाठ है जिसका उपयोग केवल सामग्री की संरचना दिखाने के लिए किया जाता है। इसका किसी वास्तविक अर्थ से कोई संबंध नहीं होता, लेकिन यह पढ़ने में स्वाभाविक लगता है। इस प्रकार के पाठ का प्रयोग आमतौर पर वेबसाइट डिज़ाइन, प्रिंट लेआउट, पोस्टर, ब्रोशर और अन्य दृश्य प्रस्तुतियों में किया जाता है ताकि वास्तविक सामग्री आने से पहले स्थान और प्रवाह को समझा जा सके

इस डमी पाठ का उद्देश्य यह दिखाना है कि अंतिम सामग्री कैसी दिखाई देगी। इसमें शब्दों की लंबाई, वाक्यों की संरचना और पैराग्राफ का संतुलन शामिल होता है। डिजाइनर और डेवलपर इसका उपयोग यह जांचने के लिए करते हैं कि टेक्स्ट फ़ॉन्ट, साइज और स्पेसिंग के साथ कैसा दिखेगा। इससे यह सुनिश्चित होता है कि वास्तविक सामग्री जोड़ने पर कोई दृश्य समस्या न हो।

हिंदी लिप्सम विशेष रूप से तब उपयोगी होता है जब प्रोजेक्ट हिंदी या देवनागरी लिपि में हो। इससे यह समझना आसान हो जाता है कि भाषा के अनुसार डिज़ाइन कितना प्रभावी है। इस तरह का पाठ न तो पाठक को विचलित करता है और न ही ध्यान वास्तविक संदेश से हटाता है, क्योंकि इसका कोई वास्तविक संदेश होता ही नहीं।

संक्षेप में, यह केवल एक भराव सामग्री है जो डिज़ाइन प्रक्रिया को आसान और अधिक प्रभावी बनाती है।

\end{hindi}

\pagebreak




